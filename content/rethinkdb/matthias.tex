\section{Limits}

As every existing technique RethinkDB, too, has restictions. 
These can be hard or soft limitations depending on their context.
\cite{RethinkLimits}

It is not possible to create more than 64 Shards in RethinkDB.
The number of databases are not limited by anything than diskspace and RAM as is the number of tables.
The recommended size limit for a single Document is 16MB and should be respected due to potential perforance issues.
Also do Arrays in a RethinkDB Server have a limited length.
This length can be configured per query and by default is at 100.000 elements.
Primary keys are limited to 127 characters.
But secondary keys aren't hard limited by those 127 characters, if they exceeds their limit, RethinkDB will use linear search for the following characters.

JSON queries are hard limited as well.
They can only be 64 MB in size or smaller.

The ordering in RethinkDB is byte-wise.
The functions orderBy and between uses the byte representaiton of the carater.
This has the result that RethinkDB dose not normalize identical characters with multiple codepoints.
For example the character "é" has the UTF-8 representation \code{\textbackslash{}u0065\textbackslash{}u0301} and \code{\textbackslash{}u00e9}, the will be grouped separately.

The usage of the cli option \code{--direct-io} is restricted to supporting file systems only.
Typically encrypted and compressed file systems will not support this option.

Numbers are double precision IEEE 754 floating point.
Integers are stored precisely from -2\textsuperscript{53} to 2\textsuperscript{53}, outside that range they will be rounded.

\section{Data Types}

RetinkDB has all the basic Data types. 
Number, string, boolean, object, array and the \code{null} value. 
Additionaly RethinkDB stores specific data types like tables, streams, selections, binary objects, time objects, geometry data types, and grouped data.
\cite{RethinkDataTypes}

\textbf{Numbers} can bee Integer or Floatingpoint values. 
They will be stored with 64-bit percision.
\code{NaN} or Infinit can not be stored.

\textbf{String} are stored with UTF-8 encoding.

\textbf{Booleans} can be \code{true} or \code{false}

\textbf{Null} is a special value.
It is not the number zero or an empty string.
It symbolizes the absence of any other value.

\textbf{Objects} are key-value pairs.
Any valid JSON object is a valid RethinkDB object.
\begin{lstlisting}[frame=single, caption=RethinkDB's object notation, label=refdoc]
{
	"key"  : "valueString",
	"key2" : false,
}
\end{lstlisting}
The keys can only be strings, but the values can be any data type.
A RethinkDB Document is a object.

\textbf{Arrays} are lists of values.
Arrays can be empty.
It is not enforced to use only on type of values in an Array.
But it is highly recommended.
\begin{lstlisting}[frame=single, caption=RethinkDB's array notation, label=refdoc]
[
	"valueString",
	false,
]
[]
\end{lstlisting}
The values of an Array can be any data type.
If you use arrays for many values, more than 100.000 you should notice that arrays are fully loaded into Server RAM before being send to the client.

Specific Data types

\textbf{Databases} are returned after a \code{db} function call.

\textbf{Tables} are returned after a \code{getAll} function call.
Their behavior is similar to selections.

\textbf{Selections} are the result of \code{filter} ot \code{get} function calls.
Selections comes in three variants \code{Selection<Object>}, \code{Selection<Array>} and \code{Selection<Stream>}.

\textbf{Streams} are like arrays, lists.
Streams are not writable and are lazy, they give you the next dataset if the current has been processed.
With Streams you can read tremendous long lists of Data, without holding everything in Memory.


\section{Changefeeds}

Changefeeds are the main part of RethinkDB's real-time functionality.
Almost all queries can be used as a changefeed.
With a Changefeed you recive any changes on your query.
\cite{RethinkChangefeeds}

\begin{lstlisting}[frame=single, caption=implementing changefeed, label=refdoc]
r.table('project').changes().run(conn, (err, cursor) => {
	cursor.each((err, row) => {
		if(err) throw err;
		processRow(row);
	})
});
\end{lstlisting}

The \code{changes()} method returns a cursor.
The \code{each()} method of that cursor object iterates over each change.
If the underlaying data of the query changes the callback of \code{each()} will be called.
For example a new project will be added \code{{id:3, name: "awesome Project", priority:1}} into the project table.
The \code{each()} callback will be called with the changes of the table.

\begin{lstlisting}[frame=single, caption=changefeed's return value, label=refdoc]
{
	old_val: null,
	new_val: {id:3, name: 'awesome Project', priority:1}
}
\end{lstlisting}

If the callback for \code{each} returns false, the cursor will stop iterating.

\textbf{single document changefeeds}

The single document changefeed only notifies the subscriber if the document changes.

\begin{lstlisting}[frame=single, caption=single document changes, label=refdoc]
r.table('project').get(3).changes().run(conn, callback);
\end{lstlisting}

This would listen for changes of the project with id 3.

\textbf{Filtering Changefeeds}

The method \code{changes()} integrates with the ReQL language.
\code{changes()} can be used with mostly any outher command.

The chaning of ReQL methods can be as complex as you wish.
For Example changefeeds can be filter:

\begin{lstlisting}[frame=single, caption=filtering changefeeds, label=refdoc]
r.table('project').changes().filter(
    r.row('new_val')('priority').lt(r.row('old_val')('priority'))
)('new_val').run(conn, callback)
\end{lstlisting}

This will be notified on every change, there the priority is lower as the previous.
Chaining methods with changefeeds has some limitations.


\textbf{handling latency}

With some latency on the client side, it is possible that there will be mutliple writes into a table, before the clinet gets all changefeed events.
By default a changefeed subscriber will only get one change object.
If this is not the desired procedure the option \code{squash: false} should be used on the \code{changes()} method.
With this the change object will be send for every change.

Changefeeds do not have a delivery guarantee.

